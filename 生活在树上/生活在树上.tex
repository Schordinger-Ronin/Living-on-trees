\documentclass[12pt,a4paper]{article}

\usepackage{ctex}

% 引入常用的包
\usepackage[utf8]{inputenc} % 设置UTF-8编码
\usepackage{amsmath, amssymb} % 数学符号
\usepackage{graphicx} % 插入图片
\usepackage{hyperref} % 超链接
\usepackage{cite} % 引用
\usepackage{geometry} % 页面布局
\usepackage{array} % 表格
\usepackage{booktabs} % 表格样式
\usepackage{caption} % 图表标题

% 页面布局设置
\geometry{left=2.5cm, right=2.5cm, top=3cm, bottom=3cm}

% 文档信息
\title{测试文档}
\author{张子健}
\date{\today}

\begin{document}

\maketitle

\begin{abstract}
本篇文章用来测试github上传步骤,没有其他含义,摘录2020浙江高考满分作文作为
样本。
\end{abstract}

\section*{题目}
(2020浙江卷高考语文作文)

每个人都有自己的人生坐标,也有对未来的美好期望。家庭可能对我们有不同的预期,社会也可能会赋予我们别样的角色。
在不断变化的现实生活中,个人与家庭、社会之间的落差或错位难免会产生。对此,你有怎样的体验与思考?写一篇文章,谈谈
自己的看法。

\begin{center}
    \textbf{\Large 生活在树上}
\end{center}

现代社会以海德格尔的一句“一切实践传统都已经瓦解完了”为嚆矢。滥觞于家庭与社会传统的期望正失去它们的借鉴意义。但面对看似无垠的未来天空,我想循卡尔维诺“树上的男爵”的生活好过过早地振翮。

我们怀揣热忱的灵魂天然被赋予对超越性的追求,不屑于古旧坐标的约束,钟情于在别处的芬芳。但当这种期望流于对过去观念不假思索的批判,乃至走向虚无与达达主义时,便值得警惕了。与秩序的落差、错位向来不能为越矩的行为张本。而纵然我们已有翔实的蓝图,仍不能自持已在浪潮之巅立下了自己的沉锚。

“我的生活故事始终内嵌在那些我由之获得自身身份共同体的故事之中。”麦金太尔之言可谓切中了肯綮。人的社会性是不可祓除的,而我们欲上青云也无时无刻不在因风借力。社会与家庭暂且被我们把握为一个薄脊的符号客体,一定程度上是因为我们尚缺乏体验与阅历去支撑自己的认知。而这种偏见的傲慢更远在知性的傲慢之上。

在孜孜矻矻以求生活意义的道路上,对自己的期望本就是在与家庭与社会对接中塑型的动态过程。而我们的底料便是对不同生活方式、不同角色的觉感与体认。生活在树上的柯希莫为强盗送书,兴修水利,又维系自己的爱情。他的生活观念是厚实的,也是实践的。倘若我们在对过往借韦伯之言“祓魅”后,又对不断膨胀的自我进行“赋魅”,那么在丢失外界预期的同时,未尝也不是丢了自我。

毫无疑问,从家庭与社会角度一觇的自我有偏狭过时的成分。但我们所应摒弃的不是对此的批判,而是其批判的廉价,其对批判投诚中的反智倾向。在尼采的观念中,如果在成为狮子与孩子之前,略去了像骆驼一样背负前人遗产的过程,那其“永远重复”洵不能成立。何况当矿工诗人陈年喜顺从编辑的意愿,选择写迎合读者的都市小说,将他十六年的地底生涯降格为桥段素材时,我们没资格斥之以媚俗。

蓝图上的落差终归只是理念上的区分,在实践场域的分野也未必明晰。譬如当我们追寻心之所向时,在途中涉足权力的玉墀,这究竟是伴随着期望的泯灭还是期望的达成?在我们塑造生活的同时,生活也在浇铸我们。既不可否认原生的家庭性与社会性,又承认自己的图景有轻狂的失真,不妨让体验走在言语之前。用不被禁锢的头脑去体味切斯瓦夫·米沃什的大海与风帆,并效维特根斯坦之言,对无法言说之事保持沉默。

用在树上的生活方式体现个体的超越性,保持婞直却又不拘泥于所谓“遗世独立”的单向度形象。这便是卡尔维诺为我们提供的理想期望范式。生活在树上——始终热爱大地——升上天空。

\section*{白话文翻译}
以海德格尔一句“一切实践传统都已经瓦解完了”为开端,在现代社会中,源于家庭与社会的人生标准早已失去了借鉴意义。我想,过早地向着看似无垠的未来天空“振翅飞翔”,不如像卡尔维诺笔下“树上的男爵”一般生活。

人类天生追求超越与创新,不屑于被古旧的传统束缚,而开拓崭新的生活。但当这种追求流于对过去观念不加思索的批评时,便值得警惕了。?与秩序的落差、错位不能为越矩的行为埋下伏笔。纵然我们已经有了翔实的蓝图,依然不能自恃已在浪潮之巅立下了自己的沉锚。

“我的生活故事始终内嵌在那些我用以获得自身身份共同体的故事之中。”麦金太尔之言可谓切中了要害。人的社会性是不可消除的,我们欲上青云也无时无刻不在因风借力。家庭与社会暂时被认为仅仅是一种无用的禁锢,一定程度上是因为我们还没有足够的阅历去改变我们的认知,而这种感性认识的偏见更远在理性认识的偏见之上。

在不懈寻找生活意义的道路上,我们人生坐标的确定本就是在与家庭和社会的对接中调整的过程。调整的依据便是对不同生活方式、不同角色的感觉与体认。生活在树上的柯希莫为强盗送书,兴修水利,又维系自己的爱情。他的生活观念是厚实的,也是实践的。倘若我们消除了过往的神圣后,又不断抬高自己以至于过分膨胀,那么在丢失外界预期的同时,也未尝不是丢了自我。

毫无疑问,从家庭和社会角度看待自我有狭隘和过时的成分。但我们应该摒弃的不是这种批判,而是批判的肤浅和反智倾向。在尼采看来,如果我们在成为狮子和孩子之前,跳过了像骆驼一样承载前人遗产的过程,那么“永恒重复”就无法实现。更何况,当矿工诗人陈年喜顺从编辑的意愿,选择写迎合读者的都市小说,将他十六年的地下生涯降格为桥段素材时,我们没有资格斥责他媚俗。

理想与现实的差距终究只是理念上的区分,在实际的操作中也未必清晰。比如当我们追寻内心的渴望时,在途中涉足权力的殿堂,这究竟是伴随着希望的破灭还是实现?在我们塑造生活的同时,生活也在塑造我们。既不能否认家庭和社会的影响,也要承认自己的观点有些浮夸失真,不妨让体验先于言语。用不被束缚的头脑去感受切斯瓦夫·米沃什的大海与风帆,并效仿维特根斯坦的言论,对不能言说的事保持沉默。

用在树上的生活方式来体现个人的超越性,保持独立但不拘泥于所谓“遗世独立”的单一形象。这就是卡尔维诺为我们提供的理想期望模式。生活在树上——始终热爱大地——升上天空。


\section*{点评}
“树上的男爵”:卡尔维诺小说《树上的男爵》中的人物柯西莫,柯希莫因对家庭、社会的失望而爬上了树开始了远离地面的生活,与自然界的生物建立了一种和谐的关系,他融入翁布罗萨的森林并在其中汲取力量和智慧,开拓了一个属于他自己的理想国。

文章以“树上”这一独特的视角切入,表现出对生活的一种独特见解。树上的生活象征着高远的理想和独立的精神,作者通过这一象征意义,表达了对现实生活中追求内心宁静、独立思考的向往。

文章不仅仅停留在表面的描写上,更深入探讨了生活的意义和价值,表现出作者深刻的思考和独特的见解。通过树上的生活,反映了对现实生活中浮躁、喧嚣的反思,呼吁人们追求内心的宁静和自由。


\end{document}
